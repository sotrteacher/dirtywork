\documentclass[12pt]{article}
\begin{document}
\section{Una implementaci\'{o}n de memoria compartida para xv6}
En este trabajo se describen modif\/icaciones en el c\'{o}digo fuente 
del sistema operativo xv6 para agregar una llamada al sistema con la 
que se tiene memoria compartida entre procesos de xv6. Las modif\/icaciones 
de c\'{o}digo fuente descritas en este documento est\'{a}n basadas en la 
informaci\'{o}n obtenida de la p\'{a}gina
\begin{verbatim}
https://github.com/jeffallen/xv6/commit/
ca9cd826afb140a469a5dc9e9445a534add596b8
\end{verbatim}
consultada en fecha 27 de febrero de 2020. El autor de este documento, no 
pretende ninguna autoria de las modificaciones de c\'{o}digo fuente 
aqu\'{i} presentadas, y las incluye \'{u}nicamente con prop\'{o}sitos 
acad\'{e}micos en el contexto de un curso de Sistemas operativos en tiempo 
real, impartido en la UPIITA del IPN, M\'{e}xico, durante el semestre 
enero-julio de 2020.

\subsection{Archivo {\tt defs.h}}
\begin{verbatim}
  defs.h 
  @@ -116,6 +116,8 @@ void            userinit(void);
  int             wait(void);
  void            wakeup(void*);
  void            yield(void);
+ void            sharedinit(void);
+ struct shared * sharedalloc(void);
  
\end{verbatim}

\subsection{Archivo {\tt main.c}}
\begin{verbatim}
   main.c 
    @@ -28,6 +28,7 @@ main(void)
    consoleinit();   // I/O devices & their interrupts
    uartinit();      // serial port
    pinit();         // process table
+   sharedinit();    // shared memory records
    tvinit();        // trap vectors
    binit();         // buffer cache
    fileinit();      // file table
  
\end{verbatim}
\subsection{Archivo {\tt proc.c}}
\begin{verbatim}
  proc.c 
    @@ -12,6 +12,11 @@ struct {
      struct proc proc[NPROC];
    } ptable;
    
+   struct {
+   struct spinlock lock;
+   struct shared shared[NSHARED];
+   } sharedtable;
+   
    static struct proc *initproc;
    
    int nextpid = 1;
    @@ -122,6 +127,49 @@ growproc(int n)
      return 0;
  }
  
+ void
+ sharedinit()
+ {
+   initlock(&sharedtable.lock, "shared");
+ }
+ 
+ struct shared *
+ sharedalloc()
+ {
+   int i;
+   void *mem;
+   struct shared *sh;
+ 
+ acquire(&sharedtable.lock);
+ for (i = 0; i < NSHARED; i++) {
+   if (sharedtable.shared[i].refcount == 0) {
+     // found a free one
+     break;
+   }
+ }
+ 
+   // no free shared records left
+   if (i == NSHARED) {
+     release(&sharedtable.lock);
+     return 0;
+   }
+ 
+   mem = kalloc();
+   if (!mem) {
+     release(&sharedtable.lock);
+     return 0;
+   }
+ 
+   sh = &sharedtable.shared[i];
+   sh->refcount = 1;
+   sh->page = mem;
+   release(&sharedtable.lock);
+ 
+   return sh;
+ }
+ 
+ extern void mappages(pde_t *pgdir, void *va, uint size, uint pa, int perm);
+ 
  // Create a new process copying p as the parent.
  // Sets up stack to return as if from system call.
  // Caller must set state of returned proc to RUNNABLE.
  @@ -153,6 +201,15 @@ fork(void)
      if(proc->ofile[i])
        np->ofile[i] = filedup(proc->ofile[i]);
    np->cwd = idup(proc->cwd);
+ 
+   // shared mem
+   if (proc->shared) {
+     acquire(&sharedtable.lock);
+     proc->shared->refcount++;
+     mappages(np->pgdir, (char *)SHARED_V, PGSIZE, v2p(proc->shared->page),
+     PTE_W|PTE_U);
+     np->shared = proc->shared;
+     release(&sharedtable.lock);
+   }
  
    pid = np->pid;
    np->state = RUNNABLE;
  @@ -230,6 +287,15 @@ wait(void)
          p->parent = 0;
          p->name[0] = 0;
          p->killed = 0;
+         if (p->shared) {
+           acquire(&sharedtable.lock);
+           p->shared->refcount--;
+           if (p->shared->refcount == 0) {
+             kfree(p->shared->page);
+         }
+         release(&sharedtable.lock);
+         p->shared = 0;
+       }
        release(&ptable.lock);
        return pid;
      }
\end{verbatim}
\subsection{Archivo {\tt proc.h}}
\begin{verbatim}
   13  proc.h 
  @@ -51,6 +51,17 @@ struct context {
  
  enum procstate { UNUSED, EMBRYO, SLEEPING, RUNNABLE, RUNNING, ZOMBIE };
  
+ // The virtual address where shared memory appears, if requested
+ #define SHARED_V 0x70000000
+ 
+ // the maximum number of shared pages in the system
+ #define NSHARED 10
+ 
+ struct shared {
+   int refcount;
+   void *page; 
+ };
+ 
  // Per-process state
  struct proc {
    uint sz;                     // Size of process memory (bytes)
  @@ -65,6 +76,7 @@ struct proc {
    int killed;                  // If non-zero, have been killed
    struct file *ofile[NOFILE];  // Open files
    struct inode *cwd;           // Current directory
+   struct shared *shared;       // Shared memory record (0 -> none)
    char name[16];               // Process name (debugging)
  };
  
  @@ -73,3 +85,4 @@ struct proc {
  //   original data and bss
  //   fixed-size stack
  //   expandable heap
+ //   (optionally) fixed-sized shared mem segment, 1 page @ 0x7000000
\end{verbatim}
\subsection{Archivo {\tt syscall.c}}
\begin{verbatim}
   2  syscall.c 
  @@ -98,6 +98,7 @@ extern int sys_unlink(void);
  extern int sys_wait(void);
  extern int sys_write(void);
  extern int sys_uptime(void);
+ extern int sys_shared(void);
  
  static int (*syscalls[])(void) = {
  [SYS_fork]    sys_fork,
  @@ -121,6 +122,7 @@ static int (*syscalls[])(void) = {
  [SYS_link]    sys_link,
  [SYS_mkdir]   sys_mkdir,
  [SYS_close]   sys_close,
+ [SYS_shared]  sys_shared,
  };
  
  void
\end{verbatim}
\subsection{Archivo {\tt syscall.h}}
\begin{verbatim}
   2  syscall.h 
  @@ -21,3 +21,5 @@
  #define SYS_link   19
  #define SYS_mkdir  20
  #define SYS_close  21
+ 
+ #define SYS_shared 22
\end{verbatim}
\subsection{Archivo {\tt sysproc.c}}
\begin{verbatim}
   24  sysproc.c 
  @@ -88,3 +88,27 @@ sys_uptime(void)
    release(&tickslock);
    return xticks;
  }
+ 
+ extern void mappages(pde_t *pgdir,void *va,uint size,uint pa,int perm);
+ 
+ // arrange for this process to have a shared memory page
+ // that will be shared with all children
+ int
+ sys_shared(void)
+ {
+   struct shared *sh;
+ 
+   // if there's already a shared page, return now
+   if (proc->shared) {
+     return SHARED_V;
+   }
+ 
+   sh = sharedalloc();
+   if (sh) {
+     proc->shared = sh;
+     mappages(proc->pgdir,(char*)SHARED_V,PGSIZE,v2p(sh->page), 
+     PTE_W|PTE_U);
+     return SHARED_V;
+   } else {
+     return 0;
+   }
+ }
\end{verbatim}
\subsection{Archivo {\tt user.h}}
\begin{verbatim}
   1  user.h 
  @@ -22,6 +22,7 @@ int getpid(void);
  char* sbrk(int);
  int sleep(int);
  int uptime(void);
+ int shared(void);
  
\end{verbatim}
\subsection{Archivo {\tt usertests.c}}
\begin{verbatim}
   38  usertests.c 
  @@ -1597,6 +1597,43 @@ fsfull()
    printf(1, "fsfull test finished\n");
  }
  
+ void
+ sharedtest()
+ {
+   int i;
+   char *sh;
+ 
+   sh = (char *)shared();
+   if (!sh) {
+     printf(2, "shared: fail\n");
+     exit();
+ }
+ 
+   if (fork() == 0) {
+     printf(1, "pid child: %d\n", getpid());
+     strcpy(sh, "hello world");
+     exit();
+   } else {
+     int pid;
+     printf(1, "pid parent: %d\n", getpid());
+     while (i < 10000) {
+       if (strcmp(sh, "hello world") == 0) {
+         printf(2, "shared: ok after %d checks\n", i);
+         break;
+       }
+       i++;
+       // yield and let the other guy run
+       sleep(0);
+     }
+     if (i == 10000) {
+       printf(2, "shared: not ok after %d checks\n", i);
+     }
+ 
+     pid = wait();
+     printf(1, "parent reaped pid %d\n", pid);
+   }
+ }
+ 
  unsigned long randstate = 1;
  unsigned int
  rand()
  @@ -1650,6 +1687,7 @@ main(int argc, char *argv[])
    bigdir(); // slow
  
    exectest();
+   sharedtest();
  
    exit();
  }
\end{verbatim}
\subsection{Archivo {\tt usys.S}}
\begin{verbatim}
   1  usys.S 
  @@ -29,3 +29,4 @@ SYSCALL(getpid)
  SYSCALL(sbrk)
  SYSCALL(sleep)
  SYSCALL(uptime)
+ SYSCALL(shared)
\end{verbatim}
\subsection{Archivo {\tt vm.c}}
\begin{verbatim}
   6  vm.c 
  @@ -67,7 +67,7 @@ walkpgdir(pde_t *pgdir, const void *va, int alloc)
  // Create PTEs for virtual addresses starting at va that refer to
  // physical addresses starting at pa. va and size might not
  // be page-aligned.
- static int
+ int
  mappages(pde_t *pgdir, void *va, uint size, uint pa, int perm)
  {
    char *a, *last;
  @@ -223,7 +223,7 @@ allocuvm(pde_t *pgdir, uint oldsz, uint newsz)
    char *mem;
    uint a;
  
 -  if(newsz >= KERNBASE)
 +  if(newsz >= SHARED_V)
      return 0;
    if(newsz < oldsz)
      return oldsz;
  @@ -281,7 +281,7 @@ freevm(pde_t *pgdir)
  
    if(pgdir == 0)
      panic("freevm: no pgdir");
 -  deallocuvm(pgdir, KERNBASE, 0);
 +  deallocuvm(pgdir, SHARED_V, 0);
    for(i = 0; i < NPDENTRIES; i++){
      if(pgdir[i] & PTE_P){
        char * v = p2v(PTE_ADDR(pgdir[i]));

\end{verbatim}
\section{Modif\/icaciones de los archivos fuente de xv6}
Siguiendo las indicaciones anteriores, se modif\/icaron los archivos\\
{\tt defs.h}\\
{\tt main.c}\\
{\tt proc.c}\\
{\tt proc.h}\\
{\tt syscall.c}\\
{\tt syscall.h}\\
{\tt sysproc.c}\\
{\tt user.h}\\
{\tt usertests.c}\\
{\tt usys.S}\\
{\tt vm.c}\\
de la siguiente forma
\subsection{Modif\/icaciones del archivo {\tt defs.h}}
\begin{verbatim}
#define SOTR_20202_1
struct buf;
struct context;
struct file;
//...
void            wakeup(void*);
void            yield(void);
#ifdef SOTR_20202_1
void            sharedinit(void);
struct shared * sharedalloc(void);
#endif /*SOTR_20202_1*/

// swtch.S
void            swtch(struct context**, struct context*);
\end{verbatim}
\subsection{Modif\/icaciones del archivo {\tt main.c}}
\begin{verbatim}
  uartinit();      // serial port
  pinit();         // process table
#ifdef SOTR_20202_1
  sharedinit();     // shared memory records
#endif /*SOTR_20202_1*/
  tvinit();        // trap vectors
  binit();         // buffer cache
\end{verbatim}
\subsection{Modif\/icaciones del archivo {\tt proc.c}}
\begin{verbatim}
struct {
  struct spinlock lock;
  struct proc proc[NPROC];
} ptable;

#ifdef SOTR_20202_1
struct {
  struct spinlock lock;
  struct shared shared[NSHARED];
} sharedtable;
#endif /*SOTR_20202_1*/

static struct proc *initproc;

int nextpid = 1;

//...
  curproc->sz = sz;
  switchuvm(curproc);
  return 0;
}

#ifdef SOTR_20202_1
void
sharedinit()
{
  initlock(&sharedtable.lock,"shared");
}/*end void sharedinit()*/

struct shared *
sharedalloc()
{
  int i;
  void *mem;
  struct shared *sh;

  acquire(&sharedtable.lock);
  for(i=0;i<NSHARED;i++){
    if(sharedtable.shared[i].refcount==0){
      // found a free one
      break;
    }
  }

  // if no free shared records left
  if(i==NSHARED){
    release(&sharedtable.lock);
    return 0;
  }

  mem=kalloc();
  if(!mem){
    release(&sharedtable.lock);
    return 0;
  }

  sh=&sharedtable.shared[i];
  sh->refcount=1;
  sh->page=mem;
  release(&sharedtable.lock);
//cprintf("initialization of shared memory has completed\n");

  return sh;
}/*end struct shared *sharedalloc()*/

extern void mappages(pde_t *pgdir,void *va,uint size,uint pa,int perm);

#endif /*SOTR_20202_1*/

// Create a new process copying p as the parent.
// Sets up stack to return as if from system call.
// Caller must set state of returned proc to RUNNABLE.
int
fork(void)
{
  int i, pid;
  struct proc *np;
  struct proc *curproc = myproc();

  // Allocate process.
  if((np = allocproc()) == 0){
    return -1;
  }

  // Copy process state from proc.
  if((np->pgdir = copyuvm(curproc->pgdir, curproc->sz)) == 0){
    kfree(np->kstack);
    np->kstack = 0;
    np->state = UNUSED;
    return -1;
  }
  np->sz = curproc->sz;
  np->parent = curproc;
  *np->tf = *curproc->tf;

  // Clear %eax so that fork returns 0 in the child.
  np->tf->eax = 0;

  for(i = 0; i < NOFILE; i++)
    if(curproc->ofile[i])
      np->ofile[i] = filedup(curproc->ofile[i]);
  np->cwd = idup(curproc->cwd);

#ifdef SOTR_20202_1
  // shared mem
  if(curproc->shared){
    acquire(&sharedtable.lock);
    curproc->shared->refcount++;
    mappages(np->pgdir,(char*)SHARED_V,PGSIZE,V2P(curproc->shared->page),
    PTE_W|PTE_U);
    np->shared=curproc->shared;
    release(&sharedtable.lock);
  }
#endif /*SOTR_20202_1*/

  safestrcpy(np->name, curproc->name, sizeof(curproc->name));

  pid = np->pid;

  acquire(&ptable.lock);

  np->state = RUNNABLE;

  release(&ptable.lock);

  return pid;
}

//...
// Wait for a child process to exit and return its pid.
// Return -1 if this process has no children.
int
wait(void)
{
  struct proc *p;
  int havekids, pid;
  struct proc *curproc = myproc();
  
  acquire(&ptable.lock);
  for(;;){
    // Scan through table looking for exited children.
    havekids = 0;
    for(p = ptable.proc; p < &ptable.proc[NPROC]; p++){
      if(p->parent != curproc)
        continue;
      havekids = 1;
      if(p->state == ZOMBIE){
        // Found one.
        pid = p->pid;
        kfree(p->kstack);
        p->kstack = 0;
        freevm(p->pgdir);
        p->pid = 0;
        p->parent = 0;
        p->name[0] = 0;
        p->killed = 0;
#ifdef SOTR_20202_1
        if(p->shared){
          acquire(&sharedtable.lock);
          p->shared->refcount--;
          if(p->shared->refcount==0){
            kfree(p->shared->page);
          }
          release(&sharedtable.lock);
          p->shared=0;
        }
#endif /*SOTR_20202_1*/
        p->state = UNUSED;
        release(&ptable.lock);
        return pid;
      }
    }

    // No point waiting if we don't have any children.
    if(!havekids || curproc->killed){
      release(&ptable.lock);
      return -1;
    }

    // Wait for children to exit.  (See wakeup1 call in proc_exit.)
    sleep(curproc, &ptable.lock);  //DOC: wait-sleep
  }
}


\end{verbatim}
\subsection{Modif\/icaciones del archivo {\tt proc.h}}
\begin{verbatim}
#define SOTR_20202_1
// Per-CPU state
struct cpu {
  uchar apicid;                // Local APIC ID
//...
enum procstate { UNUSED, EMBRYO, SLEEPING, RUNNABLE, RUNNING, ZOMBIE };

#ifdef SOTR_20202_1
// The virtual address where shared memory appears, if requested 
#define SHARED_V	0x70000000

// The maximun number of shared pages in the system
#define NSHARED	10

struct shared {
  int refcount;
  void *page;
};
#endif /*SOTR_20202_1*/

// Per-process state
struct proc {
  uint sz;                     // Size of process memory (bytes)
  pde_t* pgdir;                // Page table
  char *kstack;                // Bottom of kernel stack for this process
  enum procstate state;        // Process state
  int pid;                     // Process ID
  struct proc *parent;         // Parent process
  struct trapframe *tf;        // Trap frame for current syscall
  struct context *context;     // swtch() here to run process
  void *chan;                  // If non-zero, sleeping on chan
  int killed;                  // If non-zero, have been killed
  struct file *ofile[NOFILE];  // Open files
  struct inode *cwd;           // Current directory
#ifdef SOTR_20202_1
  struct shared *shared;       // Shared memory record (0 -> none)
#endif /*SOTR_20202_1*/
  char name[16];               // Process name (debugging)
};

// Process memory is laid out contiguously, low addresses first:
//   text
//   original data and bss
//   fixed-size stack
//   expandable heap
#ifdef SOTR_20202_1
//   (optionally) fixed-sized shared mem segment, 1 page @ 0x70000000
#endif /*SOTR_20202_1*/
\end{verbatim}
\subsection{Modif\/icaciones del archivo {\tt syscall.c}}
\begin{verbatim}
extern int sys_uptime(void);
extern int sys_dummy(void);
#ifdef SOTR_20202_1
extern int sys_shared(void);
#endif /*SOTR_20202_1*/

static int (*syscalls[])(void) = {
[SYS_fork]    sys_fork,
[SYS_exit]    sys_exit,
[SYS_wait]    sys_wait,
[SYS_pipe]    sys_pipe,
[SYS_read]    sys_read,
[SYS_kill]    sys_kill,
[SYS_exec]    sys_exec,
[SYS_fstat]   sys_fstat,
[SYS_chdir]   sys_chdir,
[SYS_dup]     sys_dup,
[SYS_getpid]  sys_getpid,
[SYS_sbrk]    sys_sbrk,
[SYS_sleep]   sys_sleep,
[SYS_uptime]  sys_uptime,
[SYS_open]    sys_open,
[SYS_write]   sys_write,
[SYS_mknod]   sys_mknod,
[SYS_unlink]  sys_unlink,
[SYS_link]    sys_link,
[SYS_mkdir]   sys_mkdir,
[SYS_close]   sys_close,
[SYS_dummy]   sys_dummy,
#ifdef SOTR_20202_1
[SYS_shared]  sys_shared,
#endif /*SOTR_20202_1*/
};
\end{verbatim}
\subsection{Modif\/icaciones del archivo {\tt syscall.h}}
\begin{verbatim}
#define SYS_close  21
#define SYS_dummy  22
#ifdef SOTR_20202_1
#define SYS_shared  23
#endif /*SOTR_20202_1*/
\end{verbatim}
\subsection{Modif\/icaciones del archivo {\tt sysproc.c}}
\begin{verbatim}
// return how many clock tick interrupts have occurred
// since start.
int
sys_uptime(void)
{
  uint xticks;

  acquire(&tickslock);
  xticks = ticks;
  release(&tickslock);
  return xticks;
}

#ifdef SOTR_20202_1
extern void mappages(pde_t *pgdir, void *va, uint size, uint pa, int perm);

// arrange for this process to have a shared memory page
// that will be shared with all children
int
sys_shared(void)
{
  struct shared *sh;
  struct proc *proc = myproc();

  // if there's already a shared page, return now
  if (proc->shared) {
    return SHARED_V;
  }

  sh = sharedalloc();
  if (sh) {
    proc->shared = sh;
    mappages(proc->pgdir, (char *)SHARED_V, PGSIZE, V2P(sh->page), PTE_W|PTE_U);
    return SHARED_V;
  } else {
    return 0;
  }
}
#endif /*SOTR_20202_1*/
\end{verbatim}
\subsection{Modif\/icaciones del archivo {\tt user.h}}
\begin{verbatim}
#define SOTR_20202_1
struct stat;
struct rtcdate;
//...
int uptime(void);
char *dummy(int);
#ifdef SOTR_20202_1
int shared(void);
#endif /*SOTR_20202_1*/

// ulib.c
int stat(const char*, struct stat*);
\end{verbatim}
\subsection{Modif\/icaciones del archivo {\tt usertests.c}}
\begin{verbatim}
    unlink(name);
    nfiles--;
  }

  printf(1, "fsfull test finished\n");
}

#ifdef SOTR_20202_1
void
sharedtest()
{
  int i=0;
  char *sh;

  sh = (char *)shared();
  if (!sh) {
    printf(2, "shared: fail\n");
    exit();
  }

  if (fork() == 0) {
    printf(1, "pid child: %d\n", getpid());
    strcpy(sh, "hello world");
    exit();
  } else {
    int pid;
    printf(1, "pid parent: %d\n", getpid());
    while (i < 10000) {
      if (strcmp(sh, "hello world") == 0) {
        printf(2, "shared: ok after %d checks\n", i);
        break;
      }
      i++;
      // yield and let the other guy run
      sleep(0);
    }
    if (i == 10000) {
      printf(2, "shared: not ok after %d checks\n", i);
    }

    pid = wait();
    printf(1, "parent reaped pid %d\n", pid);
  }
}
#endif /*SOTR_20202_1*/

void
uio()
{
  #define RTC_ADDR 0x70
  #define RTC_DATA 0x71

  ushort port = 0;
//...
int
main(int argc, char *argv[])
{
  printf(1, "usertests starting\n");

  if(open("usertests.ran", 0) >= 0){
    printf(1, "already ran user tests -- rebuild fs.img\n");
    exit();
  }
  close(open("usertests.ran", O_CREATE));

#ifndef SOTR_20202_1
  argptest();
  createdelete();
  linkunlink();
  concreate();
  fourfiles();
  sharedfd();

  bigargtest();
  bigwrite();
  bigargtest();
  bsstest();
  sbrktest();
  validatetest();

  opentest();
  writetest();
  writetest1();
  createtest();

  openiputtest();
  exitiputtest();
  iputtest();

  mem();
  pipe1();
  preempt();
  exitwait();

  rmdot();
  fourteen();
  bigfile();
  subdir();
  linktest();
  unlinkread();
  dirfile();
  iref();
  forktest();
  bigdir(); // slow

  uio();

  exectest();
#else
  sharedtest();
#endif /*SOTR_20202_1*/

  exit();
}
\end{verbatim}
\subsection{Modif\/icaciones del archivo {\tt usys.S}}
\begin{verbatim}
#define SOTR_20202_1
#include "syscall.h"
#include "traps.h"

#define SYSCALL(name) \
  .globl name; \
  name: \
    movl $SYS_ ## name, %eax; \
    int $T_SYSCALL; \
    ret

SYSCALL(fork)
SYSCALL(exit)
SYSCALL(wait)
SYSCALL(pipe)
SYSCALL(read)
SYSCALL(write)
SYSCALL(close)
SYSCALL(kill)
SYSCALL(exec)
SYSCALL(open)
SYSCALL(mknod)
SYSCALL(unlink)
SYSCALL(fstat)
SYSCALL(link)
SYSCALL(mkdir)
SYSCALL(chdir)
SYSCALL(dup)
SYSCALL(getpid)
SYSCALL(sbrk)
SYSCALL(sleep)
SYSCALL(uptime)
SYSCALL(dummy)
#ifdef SOTR_20202_1
SYSCALL(shared)
#endif /*SOTR_20202_1*/
\end{verbatim}
\subsection{Modif\/icaciones del archivo {\tt vm.c}}
\begin{verbatim}
// Create PTEs for virtual addresses starting at va that refer to
// physical addresses starting at pa. va and size might not
// be page-aligned.
#ifdef SOTR_20202_1
int
#else
static int
#endif /*SOTR_20202_1*/
mappages(pde_t *pgdir, void *va, uint size, uint pa, int perm)
{
  char *a, *last;
  pte_t *pte;

  a = (char*)PGROUNDDOWN((uint)va);
  last = (char*)PGROUNDDOWN(((uint)va) + size - 1);
  for(;;){
    if((pte = walkpgdir(pgdir, a, 1)) == 0)
      return -1;
    if(*pte & PTE_P)
      panic("remap");
    *pte = pa | perm | PTE_P;
    if(a == last)
      break;
    a += PGSIZE;
    pa += PGSIZE;
  }
  return 0;
}

// There is one page table per process, plus one that's used when
// a CPU is not running any process (kpgdir). The kernel uses the
// current process's page table during system calls and interrupts;
// page protection bits prevent user code from using the kernel's
// mappings.
//
// setupkvm() and exec() set up every page table like this:
//
//   0..KERNBASE: user memory (text+data+stack+heap), mapped to
//                phys memory allocated by the kernel
//   KERNBASE..KERNBASE+EXTMEM: mapped to 0..EXTMEM (for I/O space)
//   KERNBASE+EXTMEM..data: mapped to EXTMEM..V2P(data)
//                for the kernel's instructions and r/o data
//   data..KERNBASE+PHYSTOP: mapped to V2P(data)..PHYSTOP,
//                                  rw data + free physical memory
//   0xfe000000..0: mapped direct (devices such as ioapic)
//
// The kernel allocates physical memory for its heap and for user memory
// between V2P(end) and the end of physical memory (PHYSTOP)
// (directly addressable from end..P2V(PHYSTOP)).

//...
  return 0;
}

// Allocate page tables and physical memory to grow process from oldsz to
// newsz, which need not be page aligned.  Returns new size or 0 on error.
int
allocuvm(pde_t *pgdir, uint oldsz, uint newsz)
{
  char *mem;
  uint a;

#ifdef SOTR_20202_1
  if(newsz >= SHARED_V)
#else
  if(newsz >= KERNBASE)
#endif /*SOTR_20202_1*/
    return 0;
  if(newsz < oldsz)
    return oldsz;

  a = PGROUNDUP(oldsz);
  for(; a < newsz; a += PGSIZE){
    mem = kalloc();
    if(mem == 0){
      cprintf("allocuvm out of memory\n");
      deallocuvm(pgdir, newsz, oldsz);
      return 0;
    }
    memset(mem, 0, PGSIZE);
    if(mappages(pgdir, (char*)a, PGSIZE, V2P(mem), PTE_W|PTE_U) < 0){
      cprintf("allocuvm out of memory (2)\n");
      deallocuvm(pgdir, newsz, oldsz);
      kfree(mem);
      return 0;
    }
  }
  return newsz;
}

//...
// Free a page table and all the physical memory pages
// in the user part.
void
freevm(pde_t *pgdir)
{
  uint i;

  if(pgdir == 0)
    panic("freevm: no pgdir");
#ifdef SOTR_20202_1
  deallocuvm(pgdir, SHARED_V, 0);
#else
  deallocuvm(pgdir, KERNBASE, 0);
#endif /*SOTR_20202_1*/
  for(i = 0; i < NPDENTRIES; i++){
    if(pgdir[i] & PTE_P){
      char * v = P2V(PTE_ADDR(pgdir[i]));
      kfree(v);
    }
  }
  kfree((char*)pgdir);
}
\end{verbatim}
\subsection{Un archivo de prueba}
\begin{verbatim}
#include "types.h"
#include "user.h"

int *intA;

int main(int argc,char *argv[])
{
  int PID;

#ifdef SOTR_20202_1
  char *sh;
  sh = (char*)shared();  /*initialize shared memory*/
  if (!sh) {
    printf(2,"shared: fail\n");
    exit();
  }
  
  intA=(int*)sh;   /*define shared pointer*/
#else
  intA=(int*)malloc(sizeof(int));
#endif /*SOTR_20202_1*/

  if(argc<2){
    printf(1,"FORMA DE USO:%s <nummber>\n",argv[0]);
    exit();
  }

  *intA=atoi(argv[1]);  /*initialize shared integer*/

  PID=fork();  /*Crear un nuevo proceso (proceso hijo)*/
  if(0==PID){  /*Proceso hijo*/
    printf(1,"PID=%d -- Proceso hijo: %d\n",PID,getpid());
    //dummy1(getpid());
    (*intA)++;
  }else{       /*Proceso padre*/
    wait();
    printf(1,"PID=%d -- Proceso padre: %d\n",PID,getpid());
    //dummy1(getpid());
    (*intA)++;
    printf(1,"*intA=%d\n",*intA);
  }
  exit();
}/*end main()*/
\end{verbatim}
\end{document}
